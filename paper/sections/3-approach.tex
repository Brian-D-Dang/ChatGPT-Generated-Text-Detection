Due to the asymmetric characteristics of directional network traffic, we opted to perform analysis of source and destination traffic separately. We accomplished this by first only considering inbound and outbound traffic destined for or originating from the well-known port range (0-1023). We then split on directionally using the source or destination port containing the well-known port. Packet count and payload size from each session was then extracted and stored by port number, maintaining time ordering, to make up the dataset upon which we performed our analysis.

For each well-known port, linear regression (least squares) was then performed on the complete set of \texttt{[payload, packet]} pairs to identify line equations (in two-point format) for source and destination traffic, making up the LR Classifier (see Figure \ref{fig:processing}). Classification consisted of performing simple distance calculations from any given point to all derived lines to identify the closest line equation.

The whole process (depicted in Figure \ref{fig:model}) was done in batches consisting of one minute long intervals, offset by 30 seconds, across an hour of network traffic observed during off peak hours. We iteratively collected a minute worth of aggregated network traffic, split and stored the features of interest, fitting the linear regression models for each port with 33\% of the sampled data, and stored the best fit line for each port.  Another minute's worth of traffic was pulled, except with an offset of 30 seconds, to form the basis of the model's ``ephemeral memory'' in classification.