This work presents itself as the first step in the hierarchy of models for network traffic classification in order to ultimately normalize the every-day activity on the network and present a relative baseline to network security professionals.

Implementing the hierarchical classification system in order to increase linear differentiability will be a point of future work due to time constraints in this project. This would include a decomposition of port 80 and 443 traffic by unique IP pair, and implementing a distance metric in order to determine the appropriate depth within the hierarchy of classification models/mechanisms.

We are also keenly interested in the incorporation non-linear models within the hierarchical scheme in order to broaden the types of traffic flows that may be described \cite{PALMIERI2010737}. A similar exploratory phase of the feature space would need to be conducted in order to identify other data points to leverage.

Additionally, this work ultimately is to inform the construction of a detection system with the goal to identify conspicuous network traffic on well-known port numbers exhibiting linear features that of another port. A testing phase would undoubtedly include generating various sets of labeled network traffic (encrypted, unencrypted, varied port numbers and payload sizes) with desired features for system testing.